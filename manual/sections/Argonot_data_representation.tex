\subsection{Representation of numerical data \sven{Remove?}}

\sven{This section should really be part of a user manual, rather than the paper!}

\subsubsection{Jacobian}

\autoref{tab:jacobian} describes the sparse representation of the Jacobians of the objective and constraints of the
Bertsimas problem. The first element contains the index of the header, at the end of the vector. The elements of the
header point to the starts of the sparse gradients of the objective and the constraints. The gradients consist in the
list of the variables with respect to which the functions depend.

The header and its pointer are colored in grey. The red elements are the variable indices of the objective function's
gradient, and the green (resp. blue and yellow) elements are the variable indices of the first (resp. second and third)
constraint's gradient.

\begin{table}[htbp]
	\centering
	\vspace{1.5cm}
	\begin{tabular}{|c|c|c|c|c|c|c|c|c|c|c|c|c|c|c|c|}
	\hline
	\cellcolor{black!20}11\tikzmark{pointer} &
	\cellcolor{red!20}1\tikzmark{f} & \cellcolor{red!20}2 & \cellcolor{red!20}3 &
	\cellcolor{green!20}3\tikzmark{c1} & \cellcolor{green!20}2 & \cellcolor{green!20}1 &
	\cellcolor{blue!20}2\tikzmark{c2} & \cellcolor{blue!20}1 &
	\cellcolor{yellow!20}2\tikzmark{c3} & \cellcolor{yellow!20}1 &
	% header
	\cellcolor{black!20}1\tikzmark{header} &
	\cellcolor{black!20}4\tikzmark{pointer_c1} &
	\cellcolor{black!20}7\tikzmark{pointer_c2} &
	\cellcolor{black!20}9\tikzmark{pointer_c3} &
	\cellcolor{black!20}11 \\
	\hline
	\end{tabular}
	\vspace{1cm}
	\caption{Bertsimas problem: sparse Jacobian representation}
	\label{tab:jacobian}
	
	\begin{tikzpicture}[overlay, remember picture, shorten >=.5pt, shorten <=.5pt]
    \draw [->, yshift=0.8\baselineskip] ({pic cs:pointer}) [bend left, ] to ({pic cs:header});
    \draw [->, yshift=-0.4\baselineskip] ({pic cs:header}) [bend left] to ({pic cs:f});
    \draw [->, yshift=-0.4\baselineskip] ({pic cs:pointer_c1}) [bend left] to ({pic cs:c1});
    \draw [->, yshift=-0.4\baselineskip] ({pic cs:pointer_c2}) [bend left] to ({pic cs:c2});
    \draw [->, yshift=-0.4\baselineskip] ({pic cs:pointer_c3}) [bend left] to ({pic cs:c3});
  \end{tikzpicture}
\end{table}

\subsubsection{Hessian}

Hessian matrix of the Lagrangian $\mathcal{L}$. 

Sparse triangular representation

Illustration on Bertsimas problem. The term on row $a$ and column $b$ corresponds to
$\frac{\partial^2 \mathcal{L}}{\partial a \partial b}$.

$n = 3$ variables: $\{x_0, x_1, x_2\}$,
$\mathit{nnz} = 10$ nonzero elements.

\begin{table}[htbp]
	\centering
	\begin{tabular}{c|R|G|B|}
	& $x_0$ & $x_1$ & $x_2$ \\
	\hline
	$x_0$ & o	& o & \\
	$x_1$ &  	& o	& \\
	$x_2$ &  	& 	& \\
	\hline
	\end{tabular}
	\caption{Bertsimas problem: sparse triangular Hessian matrix}
\end{table}

\begin{table}[htbp]
	\centering
	\begin{tabular}{|c|c|c|c|}
	\hline
	0 &
	\cellcolor{red!20}1 &
	\cellcolor{green!20}3 &
	\cellcolor{blue!20}3 \\
	\hline
	\end{tabular}
	\caption{Bertsimas problem: Hessian column starts}
\end{table}

By substracting to each term the value of the previous term, we obtain the number of nonzero elements of each column:

\begin{table}[htbp]
	\centering
	\begin{tabular}{|c|c|c|c|}
	\hline
	\cellcolor{red!20}1 &
	\cellcolor{green!20}2 &
	\cellcolor{blue!20}0 \\
	\hline
	\end{tabular}
	\caption{Bertsimas problem: Hessian column dimensions}
\end{table}

\begin{table}[htbp]
	\centering
	\begin{tabular}{|c|c|c|}
	\hline
	\cellcolor{red!20}0 &
	\cellcolor{green!20}0 & \cellcolor{green!20}1 \\
	\hline
	\end{tabular}
	\caption{Bertsimas problem: Hessian row indices}
\end{table}
