\section{Introduction}

We introduce \solvername{}, the ARGOnne Nonlinear Optimization Toolkit, a solver for nonlinearly constrained optimization
problems of the form

\begin{equation}
\label{eq:NLP}
\tag{NLP}
\begin{array}{ll} \dps
\mini_x & f(x) \\
\st 	& \underline{c} \le c(x) \le \overline{c} \\
		& \underline{x} \le x \le \overline{x} \\
\end{array}
\end{equation}

where $x \in \R^n$ with bounds $\underline{x} \in (\R \cup \{-\infty\})^n$ and $\overline{x} \in (\R \cup \{\infty\})^n$.
The objective function is $f : \R^n \to \R$, and the constraint functions are $c: \R^n \to \R^m$ with bounds
$\underline{c} \in (\R \cup \{-\infty\})^m$ and $\overline{c} \in (\R \cup \{\infty\})^m$. This formulation allows
for unbounded variables, equality constraints, etc.

Our goal is to develop an extensible open-source software framework for general nonlinearly constrained optimization
that:
\begin{enumerate}
\item provides efficient and robust implementations of existing methods;
\item allows users to experiment with new algorithmic ideas;
\item provides basic features such as interfaces to modeling languages, specialized solvers and optimality tests.
\end{enumerate}

Our ultimate goal is to promote the extension of nonlinear optimization techniques to new classes of problems such
as optimization problems with equilibrium constraints (see \cite{OutKocZow:98,LuoPanRal:96,LeyLopNoc:06,Leyf:06,FletLeyf:04,FLRS:06}),
and nonlinear robust optimization (see \cite{Leyfferetal:18}).

\paragraph{Outline.} This paper is organized as follows. In Section~\ref{S:problem}, we introduce our assumptions,
briefly review different optimization problems, and discuss relevant optimality conditions. In Section~\ref{S:description},
we introduce the basic algorithmic framework that underpins nonlinear optimization solvers, and illustrate our
framework with a small number of classical examples. In Section~\ref{S:implementation}, we map this basic algorithmic framework
to our software design for \solvername{}, introduce its main classes, and show how various nonlinear optimization
methods can be implemented. Finally, we provide preliminary numerical results in Section~\ref{S:numerics}.
